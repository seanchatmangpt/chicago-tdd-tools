\chapter{Receipt Schemas and Guard Constraints}

\section{Receipt Schema Definition}

\begin{definition}[Complete Receipt Structure]
A receipt $R$ contains:

\begin{equation}
R = (h_\obs, h_\Gamma, h_\guards, h_A, h_\measure, h_t, \text{actor}, \text{timestamp}, \text{slo})
\end{equation}

where:

\begin{tabular}{|l|l|}
\hline
\textbf{Field} & \textbf{Meaning} \\
\hline
$h_\obs$ & SHA3-256 hash of observations $\obs$ \\
\hline
$h_\Gamma$ & SHA3-256 hash of candidate proposals (alternatives considered) \\
\hline
$h_\guards$ & SHA3-256 hash of applied guard set $\guards$ \\
\hline
$h_A$ & SHA3-256 hash of action result $A$ \\
\hline
$h_\measure$ & SHA3-256 hash of measurement function $\measure$ \\
\hline
$h_t$ & SHA3-256(receipt $\parallel$ $h_{t-1}$) — Merkle root linking to previous receipt \\
\hline
actor & User or system that triggered the action \\
\hline
timestamp & Unix timestamp of action execution \\
\hline
slo & Service level objective (Hot/2ns, Warm/500ms, Cold/500ms) \\
\hline
\end{tabular}
\end{definition}

\section{Receipt JSON Schema}

\begin{lstlisting}[language=json]
{
  "$schema": "http://json-schema.org/draft-07/schema#",
  "title": "Execution Receipt",
  "type": "object",
  "required": [
    "h_obs",
    "h_guards",
    "h_action",
    "h_measure",
    "merkle_root",
    "actor",
    "timestamp",
    "slo"
  ],
  "properties": {
    "h_obs": {
      "type": "string",
      "description": "SHA3-256 hash of observations",
      "pattern": "^[a-f0-9]{64}$"
    },
    "h_gamma": {
      "type": "string",
      "description": "SHA3-256 hash of candidate proposals",
      "pattern": "^[a-f0-9]{64}$"
    },
    "h_guards": {
      "type": "string",
      "description": "SHA3-256 hash of guard set",
      "pattern": "^[a-f0-9]{64}$"
    },
    "h_action": {
      "type": "string",
      "description": "SHA3-256 hash of action",
      "pattern": "^[a-f0-9]{64}$"
    },
    "h_measure": {
      "type": "string",
      "description": "SHA3-256 hash of measurement function",
      "pattern": "^[a-f0-9]{64}$"
    },
    "merkle_root": {
      "type": "string",
      "description": "Merkle root linking receipts",
      "pattern": "^[a-f0-9]{64}$"
    },
    "prev_merkle": {
      "type": "string",
      "description": "Previous receipt's merkle root",
      "pattern": "^[a-f0-9]{64}$"
    },
    "actor": {
      "type": "string",
      "description": "Actor that triggered the action",
      "minLength": 1
    },
    "timestamp": {
      "type": "integer",
      "description": "Unix timestamp",
      "minimum": 0
    },
    "slo": {
      "type": "string",
      "enum": ["hot/2ns", "warm/500ms", "cold/500ms"],
      "description": "Service level objective"
    },
    "pattern_id": {
      "type": "integer",
      "description": "Van der Aalst pattern ID",
      "minimum": 1,
      "maximum": 43
    },
    "hook_id": {
      "type": "string",
      "description": "Knowledge hook identifier"
    }
  }
}
\end{lstlisting}

\section{Guard Constraint Specifications}

\subsection{Legality Guard}

\begin{definition}[Legality Guard]
Ensures actions comply with legal and regulatory requirements.

\textbf{Specification}:
\begin{enumerate}
  \item Action must be mapped to a workflow pattern (no ad-hoc routing)
  \item Action must not violate segregation of duties (initiator $\neq$ approver)
  \item Action must comply with role-based access control (RBAC)
  \item Action must not create retroactive events (no time-travel)
\end{enumerate}

\textbf{Implementation}:
\begin{lstlisting}[language=Rust]
pub struct LegalityGuard {
    allowed_patterns: HashSet<u32>,  // YAWL pattern IDs
    rbac_rules: Vec<RbacRule>,
}

impl Guard for LegalityGuard {
    fn check(&self, action: &Action) -> Result<(), GuardViolation> {
        // Check 1: Action follows a workflow pattern
        if !self.allowed_patterns.contains(&action.pattern_id) {
            return Err(GuardViolation::IllegalityViolation(
                "Action does not follow registered pattern".to_string()
            ));
        }

        // Check 2: Segregation of duties
        if action.initiator == action.approver {
            return Err(GuardViolation::IllegalityViolation(
                "Segregation of duties violated".to_string()
            ));
        }

        // Check 3: RBAC
        for rule in &self.rbac_rules {
            rule.check(action)?;
        }

        Ok(())
    }
}
\end{lstlisting}
\end{definition}

\subsection{Budget Guard}

\begin{definition}[Budget Guard]
Ensures actions respect financial constraints and spending limits.

\textbf{Specification}:
\begin{enumerate}
  \item Individual transaction must not exceed per-transaction limit
  \item Daily spending must not exceed daily limit
  \item Monthly spending must not exceed monthly limit
  \item Quarterly spending must not exceed quarterly limit
\end{enumerate}

\textbf{Implementation}:
\begin{lstlisting}[language=Rust]
pub struct BudgetGuard {
    per_transaction_limit: f64,
    daily_limit: f64,
    monthly_limit: f64,
    quarterly_limit: f64,
    spending_tracker: SpendingTracker,
}

impl Guard for BudgetGuard {
    fn check(&self, action: &Action) -> Result<(), GuardViolation> {
        // Check 1: Per-transaction limit
        if action.amount > self.per_transaction_limit {
            return Err(GuardViolation::BudgetViolation(
                action.amount - self.per_transaction_limit
            ));
        }

        // Check 2: Daily limit
        let daily_spent = self.spending_tracker.daily_total();
        if daily_spent + action.amount > self.daily_limit {
            return Err(GuardViolation::BudgetViolation(
                daily_spent + action.amount - self.daily_limit
            ));
        }

        // Check 3: Monthly limit
        let monthly_spent = self.spending_tracker.monthly_total();
        if monthly_spent + action.amount > self.monthly_limit {
            return Err(GuardViolation::BudgetViolation(
                monthly_spent + action.amount - self.monthly_limit
            ));
        }

        // Check 4: Quarterly limit
        let quarterly_spent = self.spending_tracker.quarterly_total();
        if quarterly_spent + action.amount > self.quarterly_limit {
            return Err(GuardViolation::BudgetViolation(
                quarterly_spent + action.amount - self.quarterly_limit
            ));
        }

        Ok(())
    }
}
\end{lstlisting}
\end{definition}

\subsection{Chronology Guard}

\begin{definition}[Chronology Guard]
Ensures actions preserve temporal ordering (no retrocausation).

\textbf{Specification}:
\begin{enumerate}
  \item Action timestamp must be $\geq$ observation timestamp
  \item Action must not pre-date its dependencies
  \item Causal ordering must be preserved (if event A causes event B, then
    timestamp(A) $<$ timestamp(B))
\end{enumerate}

\textbf{Implementation}:
\begin{lstlisting}[language=Rust]
pub struct ChronologyGuard {
    current_time: fn() -> u64,
}

impl Guard for ChronologyGuard {
    fn check(&self, action: &Action) -> Result<(), GuardViolation> {
        let now = (self.current_time)();

        // Check 1: Action cannot be in future
        if action.timestamp > now {
            return Err(GuardViolation::ChronologyViolation(
                "Action timestamp is in the future".to_string()
            ));
        }

        // Check 2: Action must not pre-date dependencies
        for dep in &action.dependencies {
            if action.timestamp < dep.timestamp {
                return Err(GuardViolation::ChronologyViolation(
                    format!("Action predates dependency: {}",
                            dep.id)
                ));
            }
        }

        Ok(())
    }
}
\end{lstlisting}
\end{definition}

\subsection{Causality Guard}

\begin{definition}[Causality Guard]
Ensures actions respect causal dependencies.

\textbf{Specification}:
\begin{enumerate}
  \item All prerequisites must be satisfied before action executes
  \item Circular dependencies are forbidden
  \item Branching is allowed; convergence must synchronize
\end{enumerate}

\textbf{Implementation}:
\begin{lstlisting}[language=Rust]
pub struct CausalityGuard {
    dependency_graph: DependencyGraph,
}

impl Guard for CausalityGuard {
    fn check(&self, action: &Action) -> Result<(), GuardViolation> {
        // Check 1: All prerequisites satisfied
        for prereq in &action.prerequisites {
            if !self.dependency_graph.is_satisfied(prereq) {
                return Err(GuardViolation::CausalityViolation(
                    format!("Prerequisite not satisfied: {}",
                            prereq.id)
                ));
            }
        }

        // Check 2: No circular dependencies
        if self.dependency_graph.has_cycle(action.id) {
            return Err(GuardViolation::CausalityViolation(
                "Circular dependency detected".to_string()
            ));
        }

        Ok(())
    }
}
\end{lstlisting}
\end{definition}

\subsection{Recursion Depth Guard (Chatman Constant)}

\begin{definition}[Recursion Depth Guard]
Enforces the Chatman Constant: maximum recursion depth $\leq 8$.

\textbf{Specification}:
\begin{enumerate}
  \item All recursive operations must have depth $\leq 8$
  \item Patterns that support iteration (loops, recursion) are bounded
  \item Each loop iteration decrements a countdown variable
\end{enumerate}

\textbf{Implementation}:
\begin{lstlisting}[language=Rust]
pub struct RecursionDepthGuard {
    max_depth: u8,  // Chatman Constant = 8
}

impl Guard for RecursionDepthGuard {
    fn check(&self, action: &Action) -> Result<(), GuardViolation> {
        // Verify recursion depth
        if let Some(recursion) = &action.recursion_info {
            if recursion.depth > self.max_depth {
                return Err(GuardViolation::RecursionDepthExceeded);
            }
        }

        Ok(())
    }
}

// Enforce at compile time with const generics
pub struct ValidatedRecursion<const DEPTH: u8>
where
    // Compile-time check: DEPTH <= 8
    [(); 8 - DEPTH as usize]:,
{
    depth: u8,
}
\end{lstlisting}
\end{definition}

\section{Guard Composition}

\begin{definition}[Guard Set]
Multiple guards are composed into a GuardSet that all must pass:

\begin{equation}
\text{GuardSet} = (\text{Legality}, \text{Budget}, \text{Chronology}, \text{Causality}, \text{RecursionDepth})
\end{equation}

An action passes if all guards pass:

\begin{equation}
\text{pass} = \bigwedge_{g \in \text{GuardSet}} g(\text{action})
\end{equation}
\end{definition}

\begin{lstlisting}[language=Rust]
pub struct GuardSet {
    legality: LegalityGuard,
    budget: BudgetGuard,
    chronology: ChronologyGuard,
    causality: CausalityGuard,
    recursion: RecursionDepthGuard,
}

impl GuardSet {
    pub fn check_all(&self, action: &Action) -> Result<(), Vec<GuardViolation>> {
        let mut violations = Vec::new();

        // Check all guards, collect violations
        if let Err(e) = self.legality.check(action) {
            violations.push(e);
        }
        if let Err(e) = self.budget.check(action) {
            violations.push(e);
        }
        if let Err(e) = self.chronology.check(action) {
            violations.push(e);
        }
        if let Err(e) = self.causality.check(action) {
            violations.push(e);
        }
        if let Err(e) = self.recursion.check(action) {
            violations.push(e);
        }

        if violations.is_empty() {
            Ok(())
        } else {
            Err(violations)
        }
    }
}
\end{lstlisting}

\section{Example Receipt}

\begin{lstlisting}[language=json]
{
  "h_obs": "abc123def456...",
  "h_guards": "def789ghi012...",
  "h_action": "xyz789abc012...",
  "h_measure": "measure_fn_hash...",
  "merkle_root": "merkle_hash...",
  "prev_merkle": "prev_merkle_hash...",
  "actor": "user:alice",
  "timestamp": 1731754800,
  "slo": "hot/2ns",
  "pattern_id": 4,
  "hook_id": "hook_xor_split"
}
\end{lstlisting}

This receipt cryptographically proves that:

\begin{enumerate}
  \item Observations were received (hash $h_{obs}$)
  \item Legality, budget, chronology, and causality guards passed (hash $h_{guards}$)
  \item A specific action was executed (hash $h_{action}$)
  \item The measurement function was applied (hash $h_{measure}$)
  \item The action links to the previous receipt (merkle root)
  \item The actor was $\text{user:alice}$ (timestamp $1731754800$)
  \item Pattern 4 (Exclusive Choice) was used
  \item Execution stayed within SLO (hot path $\leq 2$ ns)
\end{enumerate}

The receipt can be independently verified by recomputing hashes and checking the Merkle chain.
