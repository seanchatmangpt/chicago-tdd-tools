\chapter{The Chatman Equation: Formal Definitions}

\section{Core Definitions}

\begin{definition}[Observation]
An observation \(\obs \in \mathcal{O}\) is a typed RDF workflow graph conforming to schema
\(\ontology \in \Sigma\). Formally, \(\obs\) is a set of RDF triples \(\{(s, p, o)\}\) where
each triple is validated against SHACL shape constraints.

\begin{equation}
\obs \models \ontology \quad \text{(observation conforms to schema)}
\end{equation}
\end{definition}

\begin{definition}[Action]
An action \(A \in \mathcal{A}\) is a realized workflow execution satisfying:
\begin{enumerate}
  \item Invariant preservation: \(\forall q \in \invariants: q(\obs) \implies q(A)\)
  \item Guard satisfaction: \(\forall g \in \guards: g(A) = \text{true}\)
  \item SLO compliance: execution time \(\leq\) specified bound
  \item Receipt generation: cryptographic proof of execution path
\end{enumerate}
\end{definition}

\begin{definition}[Measurement Function]
The measurement function \(\measure : \mathcal{O} \to \mathcal{A}\) is a deterministic
projection of typed observations into actions under guard and provenance constraints:

\begin{equation}
\measure(\obs) = \arg A \text{ s.t. } \forall q \in \invariants: q(\obs) \implies q(A)
  \text{ and } \forall g \in \guards: g(A) = \text{true}
\end{equation}

The measurement function executes workflow patterns under ingress guards \(\hooks\),
producing deterministic outcomes with cryptographic receipts.
\end{definition}

\section{Mathematical Properties}

\subsection{Determinism}

\begin{theorem}[Determinism]
The measurement function satisfies determinism:

\begin{equation}
\forall \obs_1, \obs_2 \in \mathcal{O}: \obs_1 = \obs_2 \implies \measure(\obs_1) = \measure(\obs_2)
\end{equation}

\textbf{Proof Sketch}: The measurement function is a pure function (no side effects in the
mathematical model). All computation is deterministic — guard evaluation is boolean logic,
pattern selection is deterministic routing. No randomness, no timing-dependent behavior.
Given identical inputs, identical outputs follow from functional composition.
\end{theorem}

\textbf{Implementation}: In chicago-tdd-tools, determinism is enforced at three levels:

\begin{enumerate}
  \item \textbf{Type Level}: Function signatures prohibit \code{Random}, \code{Mutex},
    or \code{Cell} in hot paths. Compiler prevents non-deterministic dependencies.

  \item \textbf{Guard Level}: All guards are pure boolean functions over typed RDF graphs.
    No temporal dependencies, no I/O operations.

  \item \textbf{Test Level}: Property-based testing with fixed seeds validates that identical
    inputs produce identical outputs across runs.
\end{enumerate}

\subsection{Idempotence}

\begin{theorem}[Idempotence]
The measurement function is idempotent:

\begin{equation}
\measure \circ \measure = \measure
\end{equation}

Equivalently, applying the measurement function twice produces the same result as applying
it once:

\begin{equation}
\measure(\measure(\obs)) = \measure(\obs)
\end{equation}

\textbf{Proof Sketch}: The measurement function produces actions that maintain invariants.
Applying it to an already-executed action (which maintains invariants) produces no additional
changes. The action is a fixed point of the measurement operator.
\end{theorem}

\textbf{Implementation}: Idempotence is validated through:

\begin{enumerate}
  \item \textbf{Snapshot Testing}: Generated snapshots are identical across multiple applications
    of the same workflow pattern.

  \item \textbf{Receipt Comparison}: Merkle roots are identical for \(\measure(\obs)\) and
    \(\measure(\measure(\obs))\).

  \item \textbf{Integration Tests}: Docker-based integration tests verify idempotence across
    real service calls.
\end{enumerate}

\subsection{Typing}

\begin{theorem}[Type Preservation]
For all observations \(\obs \in \mathcal{O}\) conforming to schema \(\ontology\),
actions \(\measure(\obs) \in \mathcal{A}\) preserve type information:

\begin{equation}
\obs \models \ontology \implies \measure(\obs) \models \ontology
\end{equation}

\textbf{Proof Sketch}: The measurement function transitions typed graphs via pattern operators.
Each operator is type-safe: it accepts inputs typed by the ontology and produces outputs
typed by the same ontology. Composition of type-safe functions is type-safe.
\end{theorem}

\subsection{Provenance}

\begin{theorem}[Receipt Verifiability]
For all actions \(A = \measure(\obs)\), a receipt \(\receipt \in \mathcal{R}\) exists
such that:

\begin{equation}
h(A) = h(\measure(\obs)) \quad \text{(receipt verifies execution path)}
\end{equation}

\textbf{Proof Sketch}: Each workflow pattern operator emits a receipt containing:
\begin{itemize}
  \item Hash of inputs (\(h(\obs)\))
  \item Hash of pattern operator used (\(h(\text{pattern ID})\))
  \item Hash of guards applied (\(h(\guards)\))
  \item Hash of outputs (\(h(A)\))
  \item Merkle link to previous receipt (for chaining)
\end{itemize}

A verifier can independently recompute \(\measure(\obs)\) and validate that the hash
matches the receipt.
\end{theorem}

\subsection{Boundedness}

\begin{theorem}[Bounded Execution]
All executions of the measurement function are bounded:

\begin{equation}
t_{\text{hot}}(\measure) \leq 8 \text{ ticks} \approx 2 \text{ ns (P99)}
\end{equation}

\begin{equation}
t_{\text{warm}}(\measure) \leq 500 \text{ ms (P99)}
\end{equation}

\begin{equation}
t_{\text{cold}}(\measure) \leq 500 \text{ ms (P99)}
\end{equation}

\textbf{Proof Sketch}: The measurement function is bounded by:

\begin{enumerate}
  \item \textbf{No Unbounded Loops}: Guard constraint (the Chatman Constant) limits recursion
    depth to \(\leq 8\). All loops have explicit termination conditions.

  \item \textbf{Finite State Space}: All knowledge graphs are finite (bound by memory).
    All workflow patterns are finite-state machines.

  \item \textbf{Timeout Enforcement}: Each operation is wrapped in a timeout primitive.
    Operators that exceed their bound are terminated and reported as errors.
\end{enumerate}
\end{theorem}

\textbf{Implementation}: Boundedness is enforced via:

\begin{enumerate}
  \item \textbf{Type-Level}: \code{SizeValidatedArray<const SIZE, const MAX>} enforces compile-time
    array bounds. Max recursion depth is a compile-time constant.

  \item \textbf{Guard-Level}: Guard constraints check that every loop iteration decrements
    a countdown variable.

  \item \textbf{Measurement-Level}: Criterion benchmarks measure actual latency. RDTSC-based
    tick counters provide sub-microsecond precision.
\end{enumerate}

\section{The Shard Law and Compositionality}

\begin{theorem}[Shard Law]
The measurement function is compositional:

\begin{equation}
\measure(\obs \sqcup \Delta) = \measure(\obs) \sqcup \measure(\Delta)
\end{equation}

where \(\sqcup\) denotes the union of typed RDF graphs.

\textbf{Proof Sketch}: Workflow patterns operate independently on disjoint parts of the
knowledge graph. Adding a change to an unrelated region (disjoint in the RDF sense) produces
an additive change to the action.
\end{theorem}

This property enables:

\begin{enumerate}
  \item \textbf{Distributed Execution}: Multiple hooks can execute in parallel on disjoint
    regions of the knowledge graph.

  \item \textbf{Incremental Updates}: When \(\Delta\obs\) arrives, only affected hooks
    re-execute. Others' results are reused.

  \item \textbf{Partial Evaluation}: Pre-computed results for \(\measure(\obs)\) are valid
    even after \(\obs\) is extended with \(\Delta\).
\end{enumerate}

\section{Guard Adjunction}

\begin{theorem}[Guard Adjunction]
The measurement function is left-adjoint to guard constraints:

\begin{equation}
\measure \dashv \guards
\end{equation}

This means: \(\measure(\obs) \in A\) satisfies \(\guards\) if and only if \(\obs \in \obs\)
is such that \(\guards(\measure(\obs)) = \text{true}\).

Formally:

\begin{equation}
\guards(\measure(\obs)) = \text{true} \iff \obs \text{ is acceptable}
\end{equation}

\textbf{Implementation}: Guards are checked at ingress:

\begin{enumerate}
  \item \textbf{Legality}: Actions must comply with regulatory requirements
  \item \textbf{Budgets}: Actions must respect financial constraints
  \item \textbf{Chronology}: Actions must preserve temporal ordering (no retrocausation)
  \item \textbf{Causality}: Actions must respect causal dependencies
\end{enumerate}

In chicago-tdd-tools, guards are:

\begin{enumerate}
  \item Defined as boolean predicates over typed RDF graphs
  \item Checked by SPARQL ASK queries or SHACL shape validation
  \item Enforced before any action executes
  \item Emitted in receipt for audit
\end{enumerate}

\section{The Chatman Constant: Bounded Regeneration}

\begin{definition}[Regeneration]
Schema drift occurs when the ontology \(\ontology_t\) changes over time. Regeneration is
the process of updating code and workflow patterns to conform to the new schema:

\begin{equation}
\measure_t(\obs) \to \measure_{t+1}(\obs) \text{ as } \ontology_t \to \ontology_{t+1}
\end{equation}
\end{definition}

\begin{theorem}[Bounded Regeneration]
Regeneration halts when schema drift is below a tolerance threshold:

\begin{equation}
\mu_{t+1}(\obs) = \mu_t(\obs) \text{ while } \drift(\Sigma) > \epsilon
\end{equation}

where the halt condition is:

\begin{equation}
\text{regeneration halts when } \drift(\Sigma) \leq \epsilon \text{ (typically } 0.5\%)
\end{equation}

and receipt delta converges:

\begin{equation}
\left| h(\measure_{t+1}(\obs)) - h(\measure_t(\obs)) \right| < 10^{-3}
\end{equation}

\textbf{Implementation}: In ggen (the projection layer), schema changes trigger code generation
across Rust, TypeScript, and Python. Each cycle produces a receipt showing the drift percentage.
When drift \(\leq 0.5\%\), regeneration halts.
\end{theorem}

\section{Formal Properties Summary}

\begin{table}[H]
\centering
\caption{Mathematical Properties of the Measurement Function}
\begin{tabular}{|l|p{3cm}|p{4cm}|}
\hline
\textbf{Property} & \textbf{Formal Statement} & \textbf{Implementation} \\
\hline
Determinism & \(\obs_1 = \obs_2 \implies \measure(\obs_1) = \measure(\obs_2)\) &
  Type-level constraints, pure functions \\
\hline
Idempotence & \(\measure \circ \measure = \measure\) &
  Snapshot testing, receipt comparison \\
\hline
Typing & \(\obs \models \ontology \implies \measure(\obs) \models \ontology\) &
  Type-safe pattern operators \\
\hline
Provenance & \(h(A) = h(\measure(\obs))\) verifiable via receipt &
  Merkle-linked receipt chains \\
\hline
Boundedness & \(t(\measure) \leq 2\) ns (hot), \(\leq 500\) ms (warm/cold) &
  Timeout enforcement, tick budgets \\
\hline
Compositionality & \(\measure(\obs \sqcup \Delta) = \measure(\obs) \sqcup \measure(\Delta)\) &
  Independent hook execution \\
\hline
Guard Adjunction & \(\measure \dashv \guards\) &
  Ingress guard checking \\
\hline
Regeneration & \(\drift(\Sigma) \leq \epsilon \implies \text{halt}\) &
  ggen projection layer \\
\hline
\end{tabular}
\end{table}
