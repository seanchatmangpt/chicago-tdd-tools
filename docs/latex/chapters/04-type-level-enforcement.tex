\chapter{Type-Level Enforcement: Compile-Time Verification}

\section{Poka-Yoke Principles in Type Design}

Poka-yoke (Japanese for "mistake-proofing") prevents errors before they occur. In chicago-tdd-tools,
Poka-yoke is realized through Rust's type system:

\begin{enumerate}
  \item \textbf{Impossible States}: Invalid states are unrepresentable in the type system
  \item \textbf{Compile-Time Enforcement}: Violations fail at compile time, not runtime
  \item \textbf{Zero-Cost}: All safety is compiled away; no runtime overhead
\end{enumerate}

\section{Type-Level AAA Pattern Enforcement}

The type state pattern enforces the Arrange-Act-Assert lifecycle at the type level:

\begin{definition}[Type State Pattern]
Test states are zero-sized types with phantom type parameters representing lifecycle phases:

\begin{lstlisting}[language=Rust]
// Sealed trait for phase markers
mod private {
    pub trait Sealed {}
}

// Marker types (zero-sized)
pub struct Arrange;
pub struct Act;
pub struct Assert;

impl private::Sealed for Arrange {}
impl private::Sealed for Act {}
impl private::Sealed for Assert {}

// Generic test state with phase tracking
pub struct TestState<Phase> {
    _phase: std::marker::PhantomData<Phase>,
    data: TestData,
}

// Implementation requires phase transitions
impl TestState<Arrange> {
    pub fn new() -> Self {
        TestState {
            _phase: PhantomData,
            data: TestData::default(),
        }
    }

    // Only Arrange phase can transition to Act
    pub fn act(self) -> TestState<Act> {
        TestState {
            _phase: PhantomData,
            data: self.data,
        }
    }
}

impl TestState<Act> {
    // Only Act phase can transition to Assert
    pub fn assert(self) -> TestState<Assert> {
        TestState {
            _phase: PhantomData,
            data: self.data,
        }
    }
}
\end{lstlisting}

\textbf{Type-Theoretic Properties}:

\begin{enumerate}
  \item \textbf{Impossible Transitions}: Code that tries to transition from Arrange to Assert
    directly fails to compile. The type system prevents it.

  \item \textbf{No Reordering}: Tests that execute Assert before Act are impossible; Rust's
    borrow checker prevents calling methods in the wrong order.

  \item \textbf{Zero-Cost}: Phase marker types are zero-sized. The Rust compiler optimizes them
    away completely. No runtime cost for type safety.
\end{enumerate}
\end{definition}

\subsection{Proof that Invalid States Are Unrepresentable}

\begin{theorem}[Invalid AAA States Are Unrepresentable]
There is no valid Rust program that:

\begin{enumerate}
  \item Calls \code{assert()} before \code{act()}
  \item Calls \code{act()} twice without \code{assert()} in between
  \item Calls methods in any order other than Arrange \(\to\) Act \(\to\) Assert
\end{enumerate}

\textbf{Proof}: The Rust type system encodes transitions as method signatures:

\begin{itemize}
  \item \code{TestState<Arrange>::act()} \(\to\) \code{TestState<Act>} — only callable on Arrange
  \item \code{TestState<Act>::assert()} \(\to\) \code{TestState<Assert>} — only callable on Act
  \item No other transitions exist; the compiler rejects invalid calls at compile time
\end{itemize}

Therefore, any code that compiles has necessarily followed Arrange \(\to\) Act \(\to\) Assert.
\end{theorem}

\section{Sealed Traits for API Control}

Sealed traits prevent external implementations that might violate invariants:

\begin{definition}[Sealed Trait Pattern]
\begin{lstlisting}[language=Rust]
// Sealed trait in private module
mod sealed {
    pub trait PhaseMarker: private::Sealed {}
}

// Public API uses sealed trait
pub trait ValidPhase: sealed::PhaseMarker {}

// Only crate can implement
impl sealed::PhaseMarker for Arrange {}
impl ValidPhase for Arrange {}

// Users cannot do this; it fails at compile time
// impl sealed::PhaseMarker for MyCustomPhase {}
// error[E0277]: the trait bound `MyCustomPhase: sealed::Sealed`
//               is not satisfied
\end{lstlisting}

\textbf{Benefit}: Sealed traits ensure the framework controls what test phases exist.
Users cannot create invalid phase types that break the invariant.
\end{definition}

\section{Generic Fixtures with Associated Types}

Chicago-tdd-tools uses Generic Associated Types (GATs) to provide flexible, type-safe fixtures:

\begin{definition}[Fixture with Associated Types]
\begin{lstlisting}[language=Rust]
pub trait TestFixture<T>: Sized {
    type Error: std::error::Error;
    type Setup: Fn() -> Result<Self, Self::Error>;
    type Teardown: Fn(&mut self) -> Result<(), Self::Error>;

    fn new() -> Result<Self, Self::Error>;
    fn setup(&mut self) -> Result<(), Self::Error>;
    fn teardown(&mut self) -> Result<(), Self::Error>;
    fn test_counter(&self) -> i64;
}

// Usage
let fixture = MyTestFixture::<DataType>::new()?;
let result = fixture.test_counter();  // Type-safe access
\end{lstlisting}

\textbf{Type Safety Properties}:

\begin{enumerate}
  \item \textbf{Error Type}: Each fixture can specify its own error type; Rust ensures
    error handling is correct.

  \item \textbf{Associated Functions}: Setup and teardown are part of the type; cannot be
    omitted or reordered.

  \item \textbf{Generic Parameter}: Different test data types T are type-checked separately;
    mixing types is impossible.
\end{enumerate}
\end{definition}

\section{Const Generics for Compile-Time Validation}

Compile-time assertions use const generics to validate properties before code runs:

\begin{definition}[Const Generic Assertions]
\begin{lstlisting}[language=Rust]
pub struct SizeValidatedArray<T, const SIZE: usize, const MAX: usize> {
    data: [T; SIZE],
}

impl<T, const SIZE: usize, const MAX: usize>
    SizeValidatedArray<T, SIZE, MAX>
where
    // Compile-time constraint: SIZE <= MAX
    [(); MAX - SIZE]:,  // This syntax ensures SIZE <= MAX
{
    pub fn new(data: [T; SIZE]) -> Self {
        SizeValidatedArray { data }
    }
}

// This compiles: SIZE (5) <= MAX (10)
let valid = SizeValidatedArray::<i32, 5, 10>::new([0; 5]);

// This fails at compile time: SIZE (20) > MAX (10)
// let invalid = SizeValidatedArray::<i32, 20, 10>::new([0; 20]);
// error: assertion failed at compile time
\end{lstlisting}

\textbf{Benefit}: Invalid array sizes are rejected at compile time, not runtime.
No panic possible for size violations.
\end{definition}

\section{The Chatman Constant: Recursion Depth Enforcement}

The Chatman Constant limits recursion depth to prevent unbounded execution:

\begin{theorem}[Chatman Constant: Max Recursion Depth = 8]
All recursive operations are bounded to a maximum depth of 8 iterations. This is enforced
via a guard constraint:

\begin{equation}
\forall \text{ recursion: depth} \leq 8
\end{equation}

\textbf{Implementation}:

\begin{lstlisting}[language=Rust]
pub struct RecursionGuard {
    depth: u8,
    max_depth: u8,
}

impl RecursionGuard {
    pub fn new(max_depth: u8) -> Result<Self, GuardViolation> {
        if max_depth > 8 {
            return Err(GuardViolation::RecursionTooDeep);
        }
        Ok(RecursionGuard {
            depth: 0,
            max_depth,
        })
    }

    pub fn enter(&mut self) -> Result<(), GuardViolation> {
        if self.depth >= self.max_depth {
            return Err(GuardViolation::RecursionDepthExceeded);
        }
        self.depth += 1;
        Ok(())
    }

    pub fn exit(&mut self) {
        if self.depth > 0 {
            self.depth -= 1;
        }
    }
}
\end{lstlisting}

The guard ensures that any recursive workflow pattern (e.g., Pattern 27: Recursion)
terminates within 8 levels.
\end{theorem}

\section{Error Handling Without Unwrap/Expect}

Chicago-tdd-tools forbids \code{.unwrap()}, \code{.expect()}, \code{panic!()}, \code{todo!()},
and \code{unimplemented!()} in production code. The type system enforces proper error handling:

\begin{definition}[Error Type Enforcement]
\begin{lstlisting}[language=Rust]
// FORBIDDEN in production (caught by CI)
let value = result.unwrap();  // Compile-time lint error

// REQUIRED: Explicit error handling
let value = match result {
    Ok(v) => v,
    Err(e) => {
        alert_warning!("Operation failed: {}", e);
        default_value
    }
};

// REQUIRED: Error propagation
fn may_fail() -> Result<Value, Error> {
    let value = operation()?;  // Use ? operator
    Ok(value)
}

// REQUIRED: If-let for optional values
let value = if let Some(v) = option {
    v
} else {
    default_value
};
\end{lstlisting}

\textbf{CI Enforcement}: Clippy lint rules deny:

\begin{lstlisting}[language=bash,numbers=none]
clippy::unwrap_used
clippy::expect_used
clippy::panic
clippy::todo
clippy::unimplemented
\end{lstlisting}

Any violation causes CI to fail. Combined with git hooks (via \code{cargo make install-hooks}),
developers are prevented from even committing violating code.
\end{definition}

\section{Logging Without Println}

The alert macro system ensures structured logging:

\begin{definition}[Alert Macros for Structured Logging]
\begin{lstlisting}[language=Rust]
// FORBIDDEN
println!("Operation completed");  // Unstructured, no severity

// REQUIRED: Use alert macros
alert_critical!("Database failed: {}", error);   // log::error!
alert_warning!("Retry attempt {}", n);           // log::warn!
alert_info!("Processing {} items", count);       // log::info!
alert_success!("Completed in {}ms", elapsed);    // log::info! (special)
alert_debug!("State: {:?}", state);              // log::debug!
\end{lstlisting}

Each alert macro maps to a \code{log} crate level with semantic meaning.
\textbf{Benefit}: Logs are structured, queryable, and can be filtered by severity.
\end{definition}

\section{Type Safety for Performance}

The \code{performance} module uses RDTSC (x86_64) for sub-microsecond measurement:

\begin{definition}[RDTSC-Based Tick Measurement]
\begin{lstlisting}[language=Rust]
#[cfg(target_arch = "x86_64")]
pub struct TickCounter {
    start: u64,
}

impl TickCounter {
    pub fn now() -> Self {
        TickCounter {
            start: rdtsc(),  // x86_64-specific assembly
        }
    }

    pub fn elapsed(&self) -> u64 {
        rdtsc() - self.start
    }
}

// Usage
let counter = TickCounter::now();
// ... operation ...
let ticks = counter.elapsed();
assert!(ticks <= 8, "Hot path exceeded budget");
\end{lstlisting}

\textbf{Type Safety}: The counter is self-bounding — it cannot measure negative time or
overflow (u64 wraps, but comparison still works).
\end{definition}

\section{Immutability-First Data Structures}

Chicago-tdd-tools defaults to immutable data:

\begin{definition}[Immutable-First Design]
\begin{lstlisting}[language=Rust]
// Default: immutable (compiler enforces)
let value = 5;
// value = 10;  // error: cannot assign to immutable

// Explicit: mutable only when needed
let mut value = 5;
value = 10;  // OK, explicitly marked

// Shared ownership: immutable by default
let data = Arc::new(data);
let data_clone = Arc::clone(&data);
// data.field = 10;  // error: cannot mutate through Arc
\end{lstlisting}

\textbf{Benefit}: Immutability prevents data races and unintended mutations.
Rust's compiler proves this statically.
\end{definition}

\section{Ownership and Borrowing Guarantees}

The type system enforces memory safety without garbage collection:

\begin{theorem}[Ownership and Borrowing]
Rust's ownership system guarantees:

\begin{enumerate}
  \item \textbf{Memory Safety}: No double-free, no use-after-free, no buffer overflows
  \item \textbf{Data Race Freedom}: Compiler prevents multiple mutable references to same data
  \item \textbf{Lifetime Safety}: References cannot outlive the data they point to
\end{enumerate}

\textbf{Implementation}: These guarantees are verified at compile time via:

\begin{itemize}
  \item Borrow checker: Tracks ownership of each value
  \item Lifetime checker: Verifies reference validity
  \item Mutability analysis: Prevents unexpected mutations
\end{itemize}
\end{theorem}

\section{Summary: Type-Level Guarantees}

Chicago-tdd-tools leverages Rust's type system to provide compile-time verification of:

\begin{center}
\begin{tabular}{|l|p{4cm}|p{4cm}|}
\hline
\textbf{Property} & \textbf{Enforcement} & \textbf{Cost} \\
\hline
AAA Pattern Order & Type state machine & Zero \\
\hline
Error Handling & No unwrap/expect/panic & Better error recovery \\
\hline
Recursion Depth & Chatman Constant (\(\leq 8\)) & Zero \\
\hline
Array Bounds & Const generics & Zero \\
\hline
Memory Safety & Ownership + borrowing & Zero \\
\hline
Data Races & Compiler rejection & Zero \\
\hline
Structured Logging & Alert macros (not println) & Better diagnostics \\
\hline
Determinism & No random, no timing-dependent code & Reproducibility \\
\hline
\end{tabular}
\end{center}

If code compiles, these properties are guaranteed. No runtime verification needed.
