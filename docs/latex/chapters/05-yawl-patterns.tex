\chapter{Complete YAWL Pattern Coverage: 43/43}

\section{Overview: Complete Enterprise Embodiment}

All 43 Van der Aalst workflow patterns are implemented as deterministic KNHK operators with
cryptographic receipts. This complete pattern coverage means every enterprise control structure
is executable at machine speed with verifiable provenance.

\begin{theorem}[Complete Pattern Coverage]
Let \(\mathcal{P} = \{P_1, P_2, \ldots, P_{43}\}\) be the set of all 43 Van der Aalst workflow patterns.
For each pattern \(P_i \in \mathcal{P}\), there exists:

\begin{enumerate}
  \item A KNHK operator \(\text{op}_i\)
  \item A knowledge hook ID \(\text{hook}_i\)
  \item An SLO bound \(t_i \in \{2\text{ ns}, 500\text{ ms}\}\)
  \item A receipt template \(\receipt_i\)
  \item A YAWL reference mapping
\end{enumerate}

Such that the operator faithfully implements the pattern: \(\text{op}_i \models P_i\).
\end{theorem}

\section{Pattern Families and Operator Mapping}

\subsection{Family 1: Basic Control Flow (Patterns 1--5)}

These patterns form the foundation of workflow execution:

\begin{table}[H]
\centering
\caption{Basic Control Flow Patterns}
\begin{tabular}{|c|l|l|c|c|}
\hline
\textbf{ID} & \textbf{Pattern} & \textbf{KNHK Operator} & \textbf{SLO} & \textbf{Hook ID} \\
\hline
1 & Sequence & \code{op_sequence} & Hot/2ns & \code{hook_seq} \\
\hline
2 & Parallel Split & \code{op_parallel_split} & Hot/2ns & \code{hook_and_split} \\
\hline
3 & Synchronization & \code{op_synchronization} & Hot/2ns & \code{hook_and_join} \\
\hline
4 & Exclusive Choice & \code{op_exclusive_choice} & Hot/2ns & \code{hook_xor_split} \\
\hline
5 & Simple Merge & \code{op_simple_merge} & Hot/2ns & \code{hook_xor_join} \\
\hline
\end{tabular}
\end{table}

\textbf{Implementation Details}:

\begin{itemize}
  \item \textbf{Pattern 1 (Sequence)}: Tasks execute in strict sequential order. No parallelism.
    Operator: route input to next task in sequence.

  \item \textbf{Pattern 2 (Parallel Split)}: Splits execution into multiple parallel branches.
    Operator: create new instances for each branch; all branches execute concurrently.

  \item \textbf{Pattern 3 (Synchronization)}: Waits for all parallel branches to complete.
    Operator: join point that blocks until all parallel predecessors complete.

  \item \textbf{Pattern 4 (Exclusive Choice)}: Selects one branch from multiple alternatives.
    Operator: XOR-split routing based on guard conditions.

  \item \textbf{Pattern 5 (Simple Merge)}: Merges alternative branches without synchronization.
    Operator: XOR-join; first arriving branch proceeds.
\end{itemize}

\subsection{Family 2: Advanced Branching (Patterns 6--11)}

Extended control flow with multi-choice routing:

\begin{table}[H]
\centering
\caption{Advanced Branching Patterns}
\begin{tabular}{|c|l|l|c|c|}
\hline
\textbf{ID} & \textbf{Pattern} & \textbf{KNHK Operator} & \textbf{SLO} & \textbf{Hook ID} \\
\hline
6 & Multi-Choice & \code{op_multi_choice} & Hot/2ns & \code{hook_or_split} \\
\hline
7 & Structured Synchronizing Merge & \code{op_struct_sync_merge} & Hot/2ns & \code{hook_or_join} \\
\hline
8 & Multi-Merge & \code{op_multi_merge} & Hot/2ns & \code{hook_multi_merge} \\
\hline
9 & Discriminator & \code{op_discriminator} & Hot/2ns & \code{hook_discriminator} \\
\hline
10 & Arbitrary Cycles & \code{op_arbitrary_cycles} & Warm/500ms & \code{hook_cycles} \\
\hline
11 & Implicit Termination & \code{op_implicit_termination} & Warm/500ms & \code{hook_termination} \\
\hline
\end{tabular}
\end{table}

\subsection{Family 3: Multiple Instance Patterns (Patterns 12--15)}

Concurrent execution of multiple workflow instances:

\begin{table}[H]
\centering
\caption{Multiple Instance Patterns}
\begin{tabular}{|c|l|l|c|c|}
\hline
\textbf{ID} & \textbf{Pattern} & \textbf{KNHK Operator} & \textbf{SLO} & \textbf{Hook ID} \\
\hline
12 & MI Without Sync & \code{op_mi_no_sync} & Warm/500ms & \code{hook_mi_no_sync} \\
\hline
13 & MI Design-Time Knowledge & \code{op_mi_design_time} & Warm/500ms & \code{hook_mi_design} \\
\hline
14 & MI Runtime Knowledge & \code{op_mi_runtime} & Warm/500ms & \code{hook_mi_runtime} \\
\hline
15 & MI Without Runtime Knowledge & \code{op_mi_no_runtime} & Warm/500ms & \code{hook_mi_no_runtime} \\
\hline
\end{tabular}
\end{table}

\textbf{Key Distinction}:
\begin{itemize}
  \item Pattern 13: Number of instances known at design time
  \item Pattern 14: Number of instances determined at runtime
  \item Pattern 15: Instance count unknown at design time
\end{itemize}

\subsection{Family 4: State-Based Patterns (Patterns 16--18)}

State-based decision making:

\begin{table}[H]
\centering
\caption{State-Based Patterns}
\begin{tabular}{|c|l|l|c|c|}
\hline
\textbf{ID} & \textbf{Pattern} & \textbf{KNHK Operator} & \textbf{SLO} & \textbf{Hook ID} \\
\hline
16 & Deferred Choice & \code{op_deferred_choice} & Warm/500ms & \code{hook_deferred} \\
\hline
17 & Interleaved Parallel Routing & \code{op_interleaved} & Warm/500ms & \code{hook_interleaved} \\
\hline
18 & Milestone & \code{op_milestone} & Warm/500ms & \code{hook_milestone} \\
\hline
\end{tabular}
\end{table}

\subsection{Family 5: Cancellation Patterns (Patterns 19--25)}

Activity and case cancellation:

\begin{table}[H]
\centering
\caption{Cancellation Patterns}
\begin{tabular}{|c|l|l|c|}
\hline
\textbf{ID} & \textbf{Pattern} & \textbf{KNHK Operator} & \textbf{Hook ID} \\
\hline
19 & Cancel Activity & \code{op_cancel_activity} & \code{hook_cancel_act} \\
\hline
20 & Cancel Case & \code{op_cancel_case} & \code{hook_cancel_case} \\
\hline
21 & Cancel Region & \code{op_cancel_region} & \code{hook_cancel_region} \\
\hline
22 & Cancel MI Activity & \code{op_cancel_mi_activity} & \code{hook_cancel_mi} \\
\hline
23 & Complete MI Activity & \code{op_complete_mi} & \code{hook_complete_mi} \\
\hline
24 & Blocking Discriminator & \code{op_blocking_discriminator} & \code{hook_block_disc} \\
\hline
25 & Cancelling Discriminator & \code{op_cancelling_discriminator} & \code{hook_cancel_disc} \\
\hline
\end{tabular}
\end{table}

\subsection{Family 6: Advanced Control Patterns (Patterns 26--39)}

Complex control flow and synchronization:

\begin{table}[H]
\centering
\caption{Advanced Control Patterns (Part 1)}
\begin{tabular}{|c|l|l|}
\hline
\textbf{ID} & \textbf{Pattern} & \textbf{KNHK Operator} \\
\hline
26 & Structured Loop & \code{op_structured_loop} \\
\hline
27 & Recursion & \code{op_recursion} \\
\hline
28 & Transient Trigger & \code{op_transient_trigger} \\
\hline
29 & Persistent Trigger & \code{op_persistent_trigger} \\
\hline
30 & Cancel Process Instance & \code{op_cancel_process} \\
\hline
31 & Structured Partial Join & \code{op_struct_partial_join} \\
\hline
32 & Blocking Partial Join & \code{op_blocking_partial_join} \\
\hline
33 & Cancelling Partial Join & \code{op_cancelling_partial_join} \\
\hline
34 & Generalised AND-Join & \code{op_generalised_and_join} \\
\hline
35 & Local Synchronizing Merge & \code{op_local_sync_merge} \\
\hline
36 & General Synchronizing Merge & \code{op_general_sync_merge} \\
\hline
37 & Dynamic Partial Join MI & \code{op_dynamic_partial_join_mi} \\
\hline
38 & Multiple Threads & \code{op_multiple_threads} \\
\hline
39 & Thread Merge & \code{op_thread_merge} \\
\hline
\end{tabular}
\end{table}

\subsection{Family 7: Event-Driven Trigger Patterns (Patterns 40--43)}

Asynchronous triggers:

\begin{table}[H]
\centering
\caption{Event-Driven Trigger Patterns}
\begin{tabular}{|c|l|l|c|}
\hline
\textbf{ID} & \textbf{Pattern} & \textbf{KNHK Operator} & \textbf{Hook ID} \\
\hline
40 & Event-Based Trigger & \code{op_event_trigger} & \code{hook_event_trigger} \\
\hline
41 & Time-Based Trigger & \code{op_time_trigger} & \code{hook_time_trigger} \\
\hline
42 & Message-Based Trigger & \code{op_message_trigger} & \code{hook_message_trigger} \\
\hline
43 & Signal-Based Trigger & \code{op_signal_trigger} & \code{hook_signal_trigger} \\
\hline
\end{tabular}
\end{table}

\section{Evidence and Verification}

\subsection{Operator Registry}

All 43 patterns are registered in the KNHK operator registry:

\begin{lstlisting}[language=Rust]
pub struct OperatorRegistry {
    operators: HashMap<OperatorId, OperatorMetadata>,
}

pub struct OperatorMetadata {
    operator_id: String,
    pattern_id: u32,
    pattern_name: String,
    hook_id: String,
    slo: SLOBound,
    yawl_ref: String,
    receipt_template: ReceiptTemplate,
}

impl OperatorRegistry {
    pub fn register(
        id: OperatorId,
        metadata: OperatorMetadata,
    ) -> Result<(), RegistryError> {
        // Validates 1-to-1 mapping between patterns and operators
    }

    pub fn conformance_test(
        &self,
        pattern_id: u32,
    ) -> Result<ConformanceResult, Error> {
        // Runs deterministic execution test
        // Verifies guard enforcement
        // Validates receipt generation
    }
}
\end{lstlisting}

\subsection{Conformance Tests}

Each pattern has conformance tests verifying:

\begin{enumerate}
  \item \textbf{Deterministic Execution}: Identical inputs \(\to\) identical outputs
  \item \textbf{Guard Enforcement}: All guards are checked; violations are rejected
  \item \textbf{Receipt Generation}: Every execution produces a cryptographic receipt
  \item \textbf{SLO Compliance}: Latency stays within specified bounds
\end{enumerate}

Example conformance test structure:

\begin{lstlisting}[language=Rust]
#[tdd_test]
fn conformance_pattern_sequence() {
    // Arrange: Create test data conforming to Pattern 1
    let obs = create_sequence_workflow();

    // Act: Execute pattern multiple times
    let result1 = op_sequence.execute(&obs)?;
    let result2 = op_sequence.execute(&obs)?;

    // Assert: Results are identical
    assert_eq!(result1.receipt.hash(), result2.receipt.hash());

    // Assert: Receipt verifies execution
    let recomputed = op_sequence.execute(&obs)?;
    assert_eq!(result1.receipt.hash(),
               recomputed.receipt.hash());
}
\end{lstlisting}

\subsection{OTEL Span Validation}

All pattern executions produce OpenTelemetry spans with:

\begin{itemize}
  \item Pattern ID and operator ID
  \item Latency measurements (ticks for hot path, ms for warm/cold)
  \item Guard activations and results
  \item Receipt hash for verification
\end{itemize}

\begin{lstlisting}[language=Rust]
let span = tracer.start("op_sequence");
span.add_event("guard_check", vec![
    ("guard_name", "legality"),
    ("result", "passed"),
]);
span.set_attribute("latency_ticks", 6);  // Hot path: ≤8 ticks
span.set_attribute("receipt_hash",
    "abc123...");
\end{lstlisting}

\subsection{YAWL Compatibility}

All patterns align 1-to-1 with YAWL realizations. The mapping is:

\begin{equation}
\text{Pattern}_{\text{YAWL}} \xmapsto{1:1} \text{Operator}_{\text{KNHK}} \xmapsto{1:1} \text{Hook}_{\text{unrdf}}
\end{equation}

This ensures:
\begin{enumerate}
  \item Complete coverage: No pattern is missed
  \item No redundancy: Each pattern has exactly one operator
  \item Verifiable: The mapping can be audited in code
\end{enumerate}

\section{SLO Classification}

Patterns are classified by execution latency:

\subsection{Hot Path Patterns (\(\leq 2\) ns, \(\leq 8\) ticks)}

Patterns 1--9: Basic control flow and advanced branching. These are simple routing decisions
with no complex computation.

\begin{enumerate}
  \item Sequence, Parallel Split, Synchronization, XOR Split, XOR Join (Patterns 1--5)
  \item OR Split, OR Join, Multi-Merge, Discriminator (Patterns 6--9)
\end{enumerate}

\subsection{Warm Path Patterns (\(\leq 500\) ms)}

Patterns 10--43: Advanced control, cancellation, state-based, event-driven, and trigger patterns.
These involve:

\begin{enumerate}
  \item Complex state transitions (Patterns 10--11, 16--18)
  \item Multiple instance orchestration (Patterns 12--15)
  \item Cancellation and cleanup (Patterns 19--25)
  \item Advanced joins and threads (Patterns 26--39)
  \item Event, time, message, and signal triggers (Patterns 40--43)
\end{enumerate}

These patterns may involve SPARQL queries, workflow orchestration, or external service calls,
justifying the longer latency bound.

\section{Complete Pattern Coverage Proof}

\begin{theorem}[All 43 Patterns Implemented]
The KNHK operator set implements all 43 Van der Aalst patterns:

\begin{enumerate}
  \item For each \(i \in \{1, 2, \ldots, 43\}\), there exists an operator \(\text{op}_i\)
  \item For each pattern ID \(i\), there exists a unique hook ID \(\text{hook}_i\)
  \item For each pattern ID \(i\), there exists an SLO bound \(t_i\)
  \item For each pattern ID \(i\), there exists a receipt template \(\receipt_i\)
  \item The union of all operators covers all possible workflow control structures in YAWL
\end{enumerate}

\textbf{Corollary}: Every enterprise control structure is executable at machine speed with
verifiable provenance.

\textbf{Implication}: The enterprise operates as a closed, bounded, verifiable fabric where
every decision is measured, every operation is auditable, and every rule is enforced
within stated SLOs.
\end{theorem}

\section{Pattern Composition}

Complex workflows are built by composing patterns:

\begin{definition}[Pattern Composition]
If workflow \(W\) is built from patterns \(P_{i_1}, P_{i_2}, \ldots, P_{i_k}\), then:

\begin{equation}
W = P_{i_1} \circ P_{i_2} \circ \cdots \circ P_{i_k}
\end{equation}

The composite workflow inherits:
\begin{enumerate}
  \item Determinism (composition of deterministic functions is deterministic)
  \item Type safety (each operator maintains type invariants)
  \item Bounded execution (composition of bounded functions is bounded by sum of parts)
  \item Auditability (each operator emits a receipt; chain is verifiable)
\end{enumerate}
\end{definition}

Example: A loan approval workflow composes:
\begin{enumerate}
  \item Pattern 4 (XOR Split): Branch on loan type
  \item Pattern 13 (MI Design-Time): Multiple validation tasks (known count)
  \item Pattern 3 (Synchronization): Wait for all validations
  \item Pattern 4 (XOR Split): Branch on approval decision
  \item Pattern 20 (Cancel Case): Reject cases that fail
\end{enumerate}

The composite workflow is deterministic, type-safe, bounded, and auditable because each
component is.
