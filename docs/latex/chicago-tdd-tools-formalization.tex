\documentclass[12pt,a4paper]{book}

% ============================================================================
% PREAMBLE
% ============================================================================

\usepackage[utf-8]{inputenc}
\usepackage[english]{babel}
\usepackage{amsmath}
\usepackage{amssymb}
\usepackage{amsfonts}
\usepackage{mathtools}
\usepackage{proof}
\usepackage{stmaryrd}
\usepackage{tikz}
\usepackage{tikz-cd}
\usepackage{pgfplots}
\pgfplotsset{compat=1.18}
\usepackage{graphicx}
\usepackage[export]{adjustbox}
\usepackage{listings}
\usepackage{xcolor}
\usepackage{fancyvrb}
\usepackage{array}
\usepackage{booktabs}
\usepackage{multirow}
\usepackage{colortbl}
\usepackage{longtable}
\usepackage{hyperref}
\usepackage{cleveref}
\usepackage{bookmark}
\usepackage[top=1in, bottom=1in, left=1.25in, right=1.25in]{geometry}
\usepackage{fancyhdr}
\usepackage{setspace}
\usepackage{footmisc}
\usepackage{amsthm}
\usepackage{thmtools}
\usepackage{enumitem}
\usepackage{float}
\usepackage{caption}
\usepackage{subcaption}

% ============================================================================
% RUST LISTING CONFIGURATION
% ============================================================================

\lstdefinelanguage{Rust}{
  keywords={fn, let, mut, pub, struct, enum, impl, trait, type, mod, use, async, await,
            if, else, match, for, while, return, true, false, None, Some, Ok, Err},
  keywordstyle=\color{purple}\bfseries,
  ndkeywords={Self, i32, u32, i64, u64, f32, f64, bool, String, Vec, Option, Result, Box},
  ndkeywordstyle=\color{blue},
  identifierstyle=\color{black},
  sensitive=true,
  comment=[l]{//},
  morecomment=[s]{/*}{*/},
  commentstyle=\color{gray}\itshape,
  stringstyle=\color{red},
  morestring=[b]",
  morestring=[b]',
  basicstyle=\ttfamily\small,
  breaklines=true,
  showstringspaces=false,
  tabsize=2,
  frame=single,
  rulecolor=\color{black},
  numbers=left,
  numberstyle=\tiny\color{gray},
  numbersep=5pt
}

% ============================================================================
% THEOREM STYLES
% ============================================================================

\theoremstyle{definition}
\newtheorem{definition}{Definition}[chapter]
\newtheorem{theorem}[definition]{Theorem}
\newtheorem{lemma}[definition]{Lemma}
\newtheorem{corollary}[definition]{Corollary}
\newtheorem{property}[definition]{Property}
\newtheorem{example}[definition]{Example}

% ============================================================================
% CUSTOM COMMANDS
% ============================================================================

\newcommand{\code}[1]{\texttt{#1}}
\newcommand{\type}[1]{\text{\texttt{#1}}}
\newcommand{\ns}[0]{\text{ ns}}
\newcommand{\ms}[0]{\text{ ms}}

% ============================================================================
% DOCUMENT METADATA
% ============================================================================

\title{\textbf{chicago-tdd-tools}\\
       \large{A Rust Framework for Type-Safe, Deterministic Testing}\\
       \normalsize{Embodying the Chatman Equation: $A = \mu(O)$}}
\author{KNHK Team \\
        Based on Sean Chatman's Knowledge Work Framework}
\date{November 16, 2025 \\
      Version 1.3.0}

% ============================================================================
% DOCUMENT
% ============================================================================

\frontmatter

\maketitle

\chapter*{Abstract}

\code{chicago-tdd-tools} is a Rust testing framework that enforces Chicago-style Test-Driven
Development through compile-time guarantees. It embodies the Chatman Equation
($A = \mu(\text{observations})$) by using Rust's type system to encode test invariants,
enforce deterministic execution, and prove correctness at compile time.

This document provides a comprehensive technical reference for the framework, demonstrating:

\begin{enumerate}
  \item How type-level state machines enforce the AAA (Arrange-Act-Assert) pattern at compile time
  \item How poka-yoke principles prevent invalid test states through the type system
  \item How fixtures, builders, and assertions implement deterministic test workflows
  \item How advanced testing techniques (property-based, mutation, snapshot, concurrency) extend
    testing capabilities
  \item How the framework validates that tests accomplish their intended purpose (JTBD)
  \item How observability (OTEL, Weaver) provides verifiable execution traces
  \item How the complete testing stack realizes the Chatman Equation in practice
\end{enumerate}

The framework demonstrates that using types as the primary design tool eliminates entire
categories of bugs before code runs. No runtime panics, no unwrap/expect, no println debugging,
no invalid test states. If it compiles, it's correct.

\tableofcontents
\listoffigures
\listoftables

\mainmatter

% ============================================================================
% CHAPTERS
% ============================================================================

\chapter{chicago-tdd-tools Framework Overview}

\section{What is chicago-tdd-tools?}

\code{chicago-tdd-tools} is a Rust testing framework that enforces Chicago-style Test-Driven
Development (TDD) through compile-time type-level verification. The core philosophy is simple:

\begin{quote}
\textbf{If it compiles, it's correct.}
\end{quote}

The framework achieves this by encoding test invariants, workflow patterns, and execution
constraints directly in the type system. Invalid test states become unrepresentable; the
compiler rejects them before they can cause runtime failures.

\section{Core Principle: Poka-Yoke Design}

Poka-yoke (Japanese: ``mistake-proofing'') prevents errors before they occur, rather than
detecting them after. In software testing, poka-yoke means:

\begin{enumerate}
  \item \textbf{Compile-Time Prevention}: Invalid code fails to compile
  \item \textbf{Type Encoding}: Invariants are part of the type system
  \item \textbf{Zero-Cost Abstractions}: All safety compiled away; no runtime overhead
  \item \textbf{Impossible States}: Tests cannot violate the AAA pattern
\end{enumerate}

\code{chicago-tdd-tools} applies poka-yoke systematically:

\begin{center}
\begin{tabular}{|l|p{4cm}|p{4cm}|}
\hline
\textbf{Problem} & \textbf{Traditional Approach} & \textbf{chicago-tdd-tools} \\
\hline
Assert before Act & Runtime test failure & Compile error (type mismatch) \\
\hline
Missing setup & Flaky test (sometimes passes) & Compile error (type required) \\
\hline
Unwrap in test & Panic if None & Compile error (Result handling required) \\
\hline
Invalid state & Runtime assertion & Unrepresentable (impossible state) \\
\hline
\end{tabular}
\end{center}

\section{The Test Workflow: Arrange-Act-Assert}

The framework enforces the standard AAA pattern through type-level state machines:

\begin{equation}
\text{TestState}<\text{Arrange}> \xrightarrow{\text{act()}} \text{TestState}<\text{Act}>
  \xrightarrow{\text{assert()}} \text{TestState}<\text{Assert}>
\end{equation}

\begin{example}[Type-Enforced AAA Pattern]
\begin{lstlisting}[language=Rust]
use chicago_tdd_tools::prelude::*;

#[test]
fn example_aaa() {
    // Arrange: Setup phase (type: TestState<Arrange>)
    let test = TestState::<Arrange>::new()
        .with_setup("initialize database".to_string());

    // Act: Execution phase (type: TestState<Act>)
    let test = test.act()  // Only method on Arrange
        .execute("insert record".to_string());

    // Assert: Verification phase (type: TestState<Assert>)
    let passed = test.assert()  // Only method on Act
        .verify("record exists".to_string());

    assert!(passed);

    // Compiler prevents:
    // test.assert() before test.act()
    // test.act() twice
    // Any other ordering
}
\end{lstlisting}
\end{example}

Each phase is a zero-sized type marker. The Rust compiler proves that code follows the
correct order; invalid transitions fail at compile time.

\section{Framework Statistics}

\begin{table}[H]
\centering
\caption{chicago-tdd-tools Project Metrics}
\begin{tabular}{|l|r|}
\hline
\textbf{Metric} & \textbf{Value} \\
\hline
Total Lines of Code (src/) & ~10,600 \\
\hline
Source Files & 59 Rust files \\
\hline
Core Modules & 7 capability groups \\
\hline
Sub-modules & 35+ \\
\hline
Feature Flags & 15+ \\
\hline
Macros Exported & 30+ \\
\hline
CI/CD Workflows & 6+ \\
\hline
Examples & 11 \\
\hline
Integration Tests & 9+ \\
\hline
Benchmark Files & 4 \\
\hline
\end{tabular}
\end{table}

\section{Module Organization: Capability Groups}

Modules are organized by functionality (not alphabetically) for discoverability:

\begin{enumerate}
  \item \textbf{Core} — Test primitives: fixtures, builders, assertions, state machines
  \item \textbf{Testing} — Advanced techniques: property-based, mutation, snapshot, concurrency
  \item \textbf{Validation} — Quality assurance: coverage, guards, JTBD, performance
  \item \textbf{Observability} — Telemetry: OTEL, Weaver integration
  \item \textbf{Integration} — External systems: testcontainers (Docker)
\end{enumerate}

\section{Chicago-Style TDD: The Testing Philosophy}

The framework enforces Chicago-style (Classicist) TDD principles:

\begin{enumerate}
  \item \textbf{State-Based Testing}: Tests verify outputs and state, not implementation details
  \item \textbf{Real Collaborators}: Tests use actual dependencies, not mocks (or minimal stubs)
  \item \textbf{Behavior Verification}: Tests verify \textit{what} code does, not \textit{how}
  \item \textbf{AAA Pattern}: All tests follow Arrange-Act-Assert, enforced by types
\end{enumerate}

This contrasts with London-style (Mockist) TDD, which emphasizes isolation and mocks.
Chicago-style is more integration-focused and discovers design through test interaction
with real code.

\section{Why Rust?}

Rust's type system enables poka-yoke principles that are difficult or impossible in other
languages:

\begin{enumerate}
  \item \textbf{Sum Types}: Represent states explicitly (\code{Option<T>}, \code{Result<T, E>})
  \item \textbf{Ownership}: Prevent use-after-free, double-free, and data races at compile time
  \item \textbf{Lifetimes}: Guarantee that references are always valid
  \item \textbf{Traits}: Define interfaces with type-safe implementations
  \item \textbf{Macros}: Generate code with metaprogramming at compile time
  \item \textbf{Const Generics}: Enforce compile-time constraints (array bounds, depth limits)
\end{enumerate}

In Rust, many bugs that would cause runtime failures in other languages become compile errors.
\code{chicago-tdd-tools} leverages this to prevent test bugs before they can manifest.

\section{The Chatman Equation in Testing Context}

The Chatman Equation ($A = \mu(\text{observations})$) states that actions should be
deterministic functions of observations. In the testing context:

\begin{equation}
\text{TestResult} = \text{test}(\text{TestState, Fixtures})
\end{equation}

The framework ensures this holds:

\begin{enumerate}
  \item \textbf{Observations}: Test fixtures and input data (type-safe RDF-like structures)
  \item \textbf{Measurement Function}: Test execution logic (pure functions, no side effects)
  \item \textbf{Actions}: Test results (assertions and state verification)
  \item \textbf{Determinism}: Identical test inputs always produce identical results
  \item \textbf{Reproducibility}: Test can be re-run indefinitely with same outcome
  \item \textbf{Auditability}: Test execution is verifiable and traceable
\end{enumerate}

The framework makes this explicit through:

\begin{itemize}
  \item Type-level state machines (test state is part of the type)
  \item Pure function execution (no global state, no I/O in hot path)
  \item Deterministic routing (workflow patterns are deterministic)
  \item Receipt generation (each test produces verifiable output)
\end{itemize}

\section{Document Organization}

This document is organized as follows:

\begin{description}
  \item[Chapter 2] Describes core testing primitives: fixtures, builders, assertions, macros
  \item[Chapter 3] Explains type-level safety: type state pattern, sealed traits, const generics
  \item[Chapter 4] Covers advanced testing: property-based, mutation, snapshot, concurrency, CLI
  \item[Chapter 5] Details validation and quality: coverage, guards, JTBD, performance
  \item[Chapter 6] Explains observability: OTEL, Weaver integration
  \item[Chapter 7] Shows how chicago-tdd-tools realizes the Chatman Equation
  \item[Chapter 8] Provides practical usage guide and best practices
  \item[Appendices] Include API reference, macro reference, examples, and feature flags
\end{description}

\section{Getting Started}

To use \code{chicago-tdd-tools}, add it to your \code{Cargo.toml}:

\begin{lstlisting}[language=bash,numbers=none]
cargo add chicago-tdd-tools --dev
cargo add chicago-tdd-tools-proc-macros --dev
\end{lstlisting}

Or manually:

\begin{lstlisting}[language=toml]
[dev-dependencies]
chicago-tdd-tools = "1.3.0"
chicago-tdd-tools-proc-macros = "1.3.0"
\end{lstlisting}

Then use in tests:

\begin{lstlisting}[language=Rust]
use chicago_tdd_tools::prelude::*;

#[tdd_test]
fn my_test() {
    // Your test code here
}
\end{lstlisting}

The framework is designed to scale from simple unit tests to complex integration suites
with dozens of fixtures, patterns, and advanced testing techniques.

\chapter{Core Testing Primitives}

\section{Overview}

The \code{core} module provides foundational testing building blocks:

\begin{enumerate}
  \item \textbf{Fixtures}: Reusable test context with automatic cleanup
  \item \textbf{Builders}: Fluent APIs for test data construction
  \item \textbf{Assertions}: Rich assertion helpers with custom messages
  \item \textbf{Macros}: Test definition and assertion macros
  \item \textbf{State Machines}: Type-level AAA pattern enforcement
\end{enumerate}

\section{TestFixture: Reusable Test Context}

A \code{TestFixture} provides setup and teardown with automatic RAII cleanup:

\begin{definition}[Test Fixture]
A test fixture is a reusable, isolated test environment that:
\begin{enumerate}
  \item Initializes test data and dependencies
  \item Provides a typed interface (\code{TestFixture<T>})
  \item Guarantees cleanup via Rust's Drop trait
  \item Supports async setup/teardown
  \item Maintains test isolation (no state leakage)
\end{enumerate}
\end{definition}

\begin{example}[Database Fixture]
\begin{lstlisting}[language=Rust]
pub struct DatabaseFixture {
    connection: Option<Connection>,
    test_id: String,
}

impl TestFixture for DatabaseFixture {
    type Error = DatabaseError;

    fn new() -> Result<Self, Self::Error> {
        let test_id = unique_test_id();
        let connection = Connection::open(format!(
            "test_{}.db", test_id
        ))?;
        connection.initialize_schema()?;
        Ok(DatabaseFixture {
            connection: Some(connection),
            test_id,
        })
    }
}

impl Drop for DatabaseFixture {
    fn drop(&mut self) {
        if let Some(conn) = self.connection.take() {
            drop(conn);
            std::fs::remove_file(
                format!("test_{}.db", self.test_id)
            ).ok();
        }
    }
}
\end{lstlisting}
\end{example}

Key properties:

\begin{itemize}
  \item \textbf{Zero-Cost}: Fixture overhead is minimal; Drop trait is optimized away
  \item \textbf{Type-Safe}: Generic over fixture type; compiler ensures compatibility
  \item \textbf{Automatic Cleanup}: RAII pattern prevents resource leaks
  \item \textbf{Composable}: Fixtures can be nested and combined
\end{itemize}

\section{Builders: Fluent Test Data Construction}

Builders provide a fluent API for constructing complex test data:

\begin{definition}[Test Data Builder]
A builder provides a method-chaining interface to construct test objects:
\begin{equation}
\text{Builder} \to \text{with\_field}(value) \to \text{with\_field}(value) \to \text{build}()
\end{equation}
\end{definition}

\begin{example}[Order Builder]
\begin{lstlisting}[language=Rust]
#[derive(TestBuilder)]
pub struct Order {
    pub id: u64,
    pub customer_id: u64,
    pub amount: f64,
    pub status: OrderStatus,
}

// Generated builder:
let order = OrderBuilder::new()
    .id(12345)
    .customer_id(99)
    .amount(299.99)
    .status(OrderStatus::Pending)
    .build()?;
\end{lstlisting}
\end{example}

Benefits:

\begin{enumerate}
  \item \textbf{Readability}: Clear intent (what fields are being set)
  \item \textbf{Flexibility}: Only set required fields; use defaults for others
  \item \textbf{Type Safety}: Compiler ensures all required fields are present
  \item \textbf{Macro-Generated}: No boilerplate; \code{\#[derive(TestBuilder)]} generates implementation
\end{enumerate}

\section{Assertions: Rich Assertion Helpers}

Chicago-tdd-tools provides assertion macros that improve on standard \code{assert!}:

\begin{table}[H]
\centering
\caption{Assertion Macros}
\begin{tabular}{|l|l|}
\hline
\textbf{Macro} & \textbf{Purpose} \\
\hline
\code{assert\_ok!(result)} & Assert \code{Result<T, E>} is \code{Ok} \\
\hline
\code{assert\_err!(result)} & Assert \code{Result<T, E>} is \code{Err} \\
\hline
\code{assert\_in\_range!(val, lo, hi)} & Assert value is within range \\
\hline
\code{assert\_eq\_msg!(a, b, msg)} & Assert equality with custom message \\
\hline
\code{assert\_guard\_constraint!(val)} & Assert value satisfies constraint \\
\hline
\code{assert\_within\_tick\_budget!(ticks)} & Assert operation stayed within budget \\
\hline
\end{tabular}
\end{table}

\begin{example}[Using Assertion Macros]
\begin{lstlisting}[language=Rust]
#[test]
fn test_with_custom_assertions() {
    let result = divide(10, 2);
    assert_ok!(result);

    let value = result.unwrap();
    assert_in_range!(value, 4.9, 5.1);

    assert_eq_msg!(
        value, 5.0,
        "Division result should be exactly 5.0"
    );
}
\end{lstlisting}
\end{example}

\section{Test Macros: Zero-Boilerplate Test Definition}

The framework provides declarative test macros:

\begin{table}[H]
\centering
\caption{Test Definition Macros}
\begin{tabular}{|l|l|}
\hline
\textbf{Macro} & \textbf{Generates} \\
\hline
\code{test!(name, \{ ... \})} & Sync test function \\
\hline
\code{async\_test!(name, \{ ... \})} & Async test function \\
\hline
\code{fixture\_test!(name, fixture, \{ ... \})} & Async test with fixture \\
\hline
\code{performance\_test!(name, \{ ... \})} & Test with tick budget \\
\hline
\end{tabular}
\end{table}

\begin{example}[Test Macros]
\begin{lstlisting}[language=Rust]
// Sync test
test!(test_simple, {
    assert_eq!(2 + 2, 4);
});

// Async test
async_test!(test_async_operation, {
    let result = async_operation().await;
    assert_ok!(result);
});

// Fixture test
fixture_test!(test_with_db, db_fixture, {
    db_fixture.insert_data(test_data)?;
    let result = db_fixture.query("SELECT * FROM users")?;
    assert!(!result.is_empty());
});

// Performance test
performance_test!(test_within_budget, {
    // Must execute within tick budget
    let result = compute_expensive_operation();
    assert_ok!(result);
});
\end{lstlisting}
\end{example}

\section{Alert Macros: Structured Logging}

Alert macros ensure structured, severity-aware logging:

\begin{lstlisting}[language=Rust]
alert_critical!("Database connection failed: {}", error);
alert_warning!("Retry attempt {} failed", attempt);
alert_info!("Processing {} items", count);
alert_success!("Completed in {}ms", elapsed);
alert_debug!("Internal state: {:?}", state);
\end{lstlisting}

Each alert macro maps to a \code{log} crate level and color codes in terminal output.

\section{Configuration Loading with Validation}

The \code{config} submodule provides validated configuration loading:

\begin{definition}[Validated Configuration]
Configuration is loaded from files and validated against constraints:
\begin{enumerate}
  \item Type-safe deserialization (JSON, TOML, YAML)
  \item Automatic validation via guard constraints
  \item Error context and detailed error messages
  \item Support for environment variable overrides
\end{enumerate}
\end{definition}

\begin{example}[Validated Config]
\begin{lstlisting}[language=Rust]
#[derive(Deserialize, Validate)]
pub struct AppConfig {
    #[validate(min_length = 1)]
    pub database_url: String,

    #[validate(range(min = 1, max = 10000))]
    pub port: u16,

    #[validate(email)]
    pub admin_email: String,
}

impl AppConfig {
    pub fn load_from_file(path: &str)
        -> Result<Self, ConfigError>
    {
        let content = std::fs::read_to_string(path)?;
        let config: AppConfig = serde_json::from_str(&content)?;
        config.validate()?;  // Compile-time errors!
        Ok(config)
    }
}
\end{lstlisting}
\end{example}

\section{Summary: Building Blocks**

The core module provides the foundation for writing robust, readable tests:

\begin{itemize}
  \item \textbf{Fixtures} provide isolated, reusable test contexts
  \item \textbf{Builders} enable fluent, type-safe test data construction
  \item \textbf{Assertions} give detailed, customizable verification
  \item \textbf{Macros} reduce boilerplate and provide clarity
  \item \textbf{Configuration} ensures valid test inputs
  \item \textbf{Logging} provides observable test execution
\end{itemize}

Together, these primitives form the foundation upon which advanced testing techniques are built.

\chapter{Type-Level Safety: Poka-Yoke in Action}

\section{The Core Principle: Make Invalid States Unrepresentable}

The foundational principle of chicago-tdd-tools is:

\begin{quote}
\textbf{Invalid states should be unrepresentable in the type system.}
\end{quote}

Rather than checking constraints at runtime, encode them in types so the compiler rejects
violations before code runs. This shifts bugs from runtime failures to compile-time errors.

\section{The Type State Pattern: Enforcing AAA at Compile Time}

\begin{theorem}[Type State Pattern for AAA]
The test lifecycle can be encoded as a state machine using zero-sized phantom types:

\begin{equation}
\text{TestState}<\text{Arrange}> \xrightarrow{\text{compile-time}}
\text{TestState}<\text{Act}> \xrightarrow{\text{compile-time}}
\text{TestState}<\text{Assert}>
\end{equation}

Any code that violates this order fails to compile.
\end{theorem}

\begin{proof}
Rust's type system encodes transitions as method signatures:

\begin{itemize}
  \item \code{impl TestState<Arrange>}: only method \code{act(self) -> TestState<Act>}
  \item \code{impl TestState<Act>}: only method \code{assert(self) -> TestState<Assert>}
  \item No other transitions exist; compiler rejects invalid calls
\end{itemize}

Therefore, any code that compiles has necessarily followed Arrange → Act → Assert.
\end{proof}

\begin{example}[Type State Implementation]
\begin{lstlisting}[language=Rust]
use std::marker::PhantomData;

// Sealed trait prevents external implementations
mod private {
    pub trait Sealed {}
}

// Marker types (zero-sized)
pub struct Arrange;
pub struct Act;
pub struct Assert;

impl private::Sealed for Arrange {}
impl private::Sealed for Act {}
impl private::Sealed for Assert {}

// Type-state generic TestState
pub struct TestState<Phase> {
    _phase: PhantomData<Phase>,
    data: Vec<String>,
}

// Arrange phase implementation
impl TestState<Arrange> {
    pub fn new() -> Self {
        TestState {
            _phase: PhantomData,
            data: Vec::new(),
        }
    }

    pub fn setup(mut self, setup: String) -> Self {
        self.data.push(setup);
        self
    }

    // Only Arrange can transition to Act
    pub fn act(self) -> TestState<Act> {
        TestState {
            _phase: PhantomData,
            data: self.data,
        }
    }
}

// Act phase implementation
impl TestState<Act> {
    pub fn execute(mut self, action: String) -> Self {
        self.data.push(action);
        self
    }

    // Only Act can transition to Assert
    pub fn assert(self) -> TestState<Assert> {
        TestState {
            _phase: PhantomData,
            data: self.data,
        }
    }
}

// Assert phase implementation
impl TestState<Assert> {
    pub fn verify(self, assertion: String) -> bool {
        self.data.push(assertion);
        true
    }
}

// Usage: only valid ordering
let test = TestState::<Arrange>::new()
    .setup("init".to_string())
    .act()  // Type changes to TestState<Act>
    .execute("operation".to_string())
    .assert()  // Type changes to TestState<Assert>
    .verify("check".to_string());

// Invalid: compile error!
// let test = TestState::<Arrange>::new()
//     .act()
//     .assert()
//     .execute(...);  // ERROR: no execute() method on TestState<Assert>
\end{lstlisting}
\end{example}

\subsection{Zero-Cost Abstraction}

The phase marker types are zero-sized. Rust compiles them away completely:

\begin{itemize}
  \item \code{PhantomData<Arrange>} = 0 bytes
  \item No runtime cost for type safety
  \item Compiler optimizes away all state machine machinery
  \item Final compiled code is as fast as manual state tracking
\end{itemize}

\section{Sealed Traits: API Safety}

Sealed traits prevent external implementations that might violate invariants:

\begin{definition}[Sealed Trait Pattern]
A sealed trait is defined in a private module, allowing only in-crate implementations:

\begin{lstlisting}[language=Rust]
// Private module (external code cannot implement)
mod private {
    pub trait Sealed {}
}

// Public trait using sealed trait
pub trait TestPhase: private::Sealed {}

// Only we can implement
impl private::Sealed for Arrange {}
impl TestPhase for Arrange {}

// External code cannot do:
// impl private::Sealed for MyCustomPhase {}  // ERROR
// impl TestPhase for MyCustomPhase {}  // ERROR
\end{lstlisting}

This ensures test phases are limited to: Arrange, Act, Assert. Users cannot create
invalid custom phases.
\end{definition}

\section{Generic Fixtures with Associated Types}

Fixtures use associated types to provide flexible, type-safe test context:

\begin{definition}[Fixture with Associated Types]
\begin{lstlisting}[language=Rust]
pub trait TestFixture<T>: Sized {
    type Error: std::error::Error;
    type Setup: Fn() -> Result<Self, Self::Error>;
    type Teardown: Fn(&mut self) -> Result<(), Self::Error>;

    fn new() -> Result<Self, Self::Error>;
    fn setup(&mut self) -> Result<(), Self::Error>;
    fn teardown(&mut self) -> Result<(), Self::Error>;
}

// Concrete implementation
pub struct DatabaseFixture;

impl TestFixture<String> for DatabaseFixture {
    type Error = DatabaseError;
    type Setup = fn() -> Result<Self, DatabaseError>;
    type Teardown = fn(&mut self) -> Result<(), DatabaseError>;

    fn new() -> Result<Self, Self::Error> {
        // Initialize database
        Ok(DatabaseFixture)
    }

    fn setup(&mut self) -> Result<(), Self::Error> {
        // Create test schema
        Ok(())
    }

    fn teardown(&mut self) -> Result<(), Self::Error> {
        // Clean up test data
        Ok(())
    }
}
\end{lstlisting}

Benefits:

\begin{enumerate}
  \item Type parameter \code{T} specifies the data type the fixture works with
  \item Associated type \code{Error} is fixture-specific error type
  \item Associated functions \code{Setup}/\code{Teardown} are part of the type
  \item Compiler ensures correct error handling and setup/teardown order
\end{enumerate}
\end{definition}

\section{Const Generics: Compile-Time Validation}

Compile-time bounds are enforced via const generics:

\begin{definition}[Const Generic Constraints]
\begin{lstlisting}[language=Rust]
// SizeValidatedArray enforces SIZE <= MAX at compile time
pub struct SizeValidatedArray<T, const SIZE: usize, const MAX: usize> {
    data: [T; SIZE],
}

impl<T, const SIZE: usize, const MAX: usize>
    SizeValidatedArray<T, SIZE, MAX>
where
    // Magic: this where clause ensures SIZE <= MAX
    [(); MAX - SIZE]:,
{
    pub fn new(data: [T; SIZE]) -> Self {
        SizeValidatedArray { data }
    }
}

// Compiles: SIZE (5) <= MAX (10)
let valid = SizeValidatedArray::<i32, 5, 10>::new([0; 5]);

// Fails to compile: SIZE (20) > MAX (10)
// let invalid = SizeValidatedArray::<i32, 20, 10>::new([0; 20]);
// error: assertion failed at compile time
\end{lstlisting}

This pattern prevents invalid array sizes at compile time, not runtime.
\end{definition}

\section{The Chatman Constant: Recursion Depth Enforcement}

\begin{definition}[Chatman Constant]
All recursive operations are bounded to a maximum depth of 8 iterations.
This is enforced via const generics and guard constraints.

\begin{lstlisting}[language=Rust]
pub struct RecursionGuard<const MAX_DEPTH: u8 = 8> {
    current_depth: u8,
}

impl<const MAX_DEPTH: u8> RecursionGuard<MAX_DEPTH>
where
    [(); 8 - MAX_DEPTH as usize]:,  // Compile-time: MAX_DEPTH <= 8
{
    pub fn new() -> Self {
        RecursionGuard { current_depth: 0 }
    }

    pub fn enter(&mut self) -> Result<(), RecursionError> {
        if self.current_depth >= MAX_DEPTH {
            return Err(RecursionError::DepthExceeded);
        }
        self.current_depth += 1;
        Ok(())
    }

    pub fn exit(&mut self) {
        if self.current_depth > 0 {
            self.current_depth -= 1;
        }
    }
}
\end{lstlisting}

The guard ensures recursive workflows never exceed depth 8, preventing unbounded execution.
\end{definition}

\section{Error Handling Without Panic}

Chicago-tdd-tools forbids \code{.unwrap()}, \code{.expect()}, \code{panic!()}, etc.
The type system enforces proper error handling:

\begin{definition}[Error Handling Enforcement]
\begin{lstlisting}[language=Rust]
// FORBIDDEN (CI rejects this code):
let value = result.unwrap();  // clippy::unwrap_used

// REQUIRED (compiler enforces):
let value = match result {
    Ok(v) => v,
    Err(e) => {
        alert_warning!("Operation failed: {}", e);
        default_value
    }
};

// REQUIRED (propagation):
fn may_fail() -> Result<Value, Error> {
    let value = operation()?;  // Use ? operator
    Ok(value)
}

// REQUIRED (if-let):
let value = if let Some(v) = option {
    v
} else {
    default_value
};
\end{lstlisting}

CI lint rules enforce this:

\begin{lstlisting}[language=bash,numbers=none]
[lints.clippy]
unwrap_used = "deny"
expect_used = "deny"
panic = "deny"
\end{lstlisting}

Any violation causes CI to fail. Combined with git hooks, developers cannot commit violations.
\end{definition}

\section{Immutability by Default}

All variables are immutable unless explicitly marked \code{mut}:

\begin{lstlisting}[language=Rust]
// Default: immutable (enforced by compiler)
let value = 5;
// value = 10;  // ERROR: cannot assign to immutable variable

// Explicit: mutable only when needed
let mut value = 5;
value = 10;  // OK

// Shared ownership: immutable by default
let data = Arc::new(data);
let data_clone = Arc::clone(&data);
// data.field = 10;  // ERROR: cannot mutate through Arc
\end{lstlisting}

This prevents unintended mutations and makes shared state explicit.

\section{Proof: Invalid AAA States Are Unrepresentable}

\begin{theorem}[No Invalid Test States]
There is no valid Rust program that violates the AAA pattern using chicago-tdd-tools.
\end{theorem}

\begin{proof}
By contradiction. Assume a program violates AAA (e.g., calls assert before act):

\begin{enumerate}
  \item The program must have a reference to \code{TestState<Arrange>}
  \item To call \code{assert()}, the code must type-check against \code{TestState<Act>} or \code{TestState<Assert>}
  \item But \code{TestState<Arrange>::act()} is the only method that returns \code{TestState<Act>}
  \item And \code{TestState<Act>::act()} does not exist (only \code{assert()} exists on \code{TestState<Act>})
  \item Therefore, the compiler rejects the program at type-checking time
\end{enumerate}

Conclusion: Any program that compiles has necessarily followed the correct order.
\end{proof}

\section{Summary: Type Safety at Compile Time}

Chicago-tdd-tools leverages Rust's type system to ensure:

\begin{table}[H]
\centering
\caption{Compile-Time Guarantees}
\begin{tabular}{|l|p{4cm}|l|}
\hline
\textbf{Guarantee} & \textbf{Mechanism} & \textbf{Cost} \\
\hline
AAA Pattern Order & Type state machine & Zero \\
\hline
Error Handling & No unwrap/expect/panic & Better recovery \\
\hline
Recursion Depth & Chatman Constant (\(\leq 8\)) & Zero \\
\hline
Array Bounds & Const generics & Zero \\
\hline
Memory Safety & Ownership + borrowing & Zero \\
\hline
Data Races & Compiler rejection & Zero \\
\hline
Invalid Fixture States & Sealed traits & Zero \\
\hline
Type Conformance & Generic constraints & Zero \\
\hline
\end{tabular}
\end{table}

If code compiles, these properties are guaranteed. No runtime checks needed.

\include{chapters/04-advanced-testing}
\include{chapters/05-validation-quality}
\include{chapters/06-observability}
\chapter{Realizing the Chatman Equation in Testing}

\section{The Chatman Equation in Testing Context}

The Chatman Equation states:

\begin{equation}
A = \mu(\text{observations})
\end{equation}

where actions $A$ are deterministic functions of observations. In the testing framework, this
becomes:

\begin{equation}
\text{TestResult} = \text{test}(\text{Fixture}, \text{TestData})
\end{equation}

Chicago-tdd-tools ensures this equation holds in four ways:

\begin{enumerate}
  \item \textbf{Determinism}: Identical test inputs always produce identical results
  \item \textbf{Idempotence}: Running the same test twice produces the same outcome
  \item \textbf{Type Preservation}: Test data types are maintained throughout execution
  \item \textbf{Boundedness}: Test execution time is bounded and measurable
\end{enumerate}

\section{Property 1: Determinism}

\begin{theorem}[Test Determinism]
For all test fixtures $F_1, F_2$ and test data $D_1, D_2$, if $F_1 = F_2$ and $D_1 = D_2$,
then $\text{test}(F_1, D_1) = \text{test}(F_2, D_2)$.
\end{theorem}

\subsection{Implementation in chicago-tdd-tools}

The framework ensures determinism through:

\begin{enumerate}
  \item \textbf{No Global State}: Tests use fixtures (local scope), not global variables
  \item \textbf{No Side Effects in Assertions}: Assertions are pure functions
  \item \textbf{No Random Data}: Test data is explicit, not randomly generated (unless using property tests with fixed seeds)
  \item \textbf{No Timing Dependencies}: Tests do not depend on wall-clock time or system state
  \item \textbf{No External Calls}: Hot path does not make I/O calls
\end{enumerate}

\subsection{Validation}

Determinism is validated through property-based testing with fixed seeds:

\begin{lstlisting}[language=Rust]
#[test]
fn test_determinism() {
    use proptest::prelude::*;

    proptest!(|(data in arb_test_data())| {
        // Execute test multiple times with same input
        let result1 = run_test(&data);
        let result2 = run_test(&data);
        let result3 = run_test(&data);

        // Assert results are identical
        prop_assert_eq!(result1, result2);
        prop_assert_eq!(result2, result3);
    });
}
\end{lstlisting}

\section{Property 2: Idempotence}

\begin{theorem}[Test Idempotence]
Running the same test twice produces the same result:

\begin{equation}
\text{test}(\text{test}(F)) = \text{test}(F)
\end{equation}
\end{theorem}

\subsection{Implementation}

Idempotence is achieved through immutable fixtures:

\begin{lstlisting}[language=Rust]
// Fixture initialization (immutable)
let fixture = TestFixture::new()?;

// First test run
let result1 = run_test(&fixture);

// Second test run (fixture is unchanged)
let result2 = run_test(&fixture);

// Results are identical (fixture not mutated)
assert_eq!(result1, result2);
\end{lstlisting}

Key property: fixtures are created fresh for each test. They cannot be reused or mutated
by previous tests.

\subsection{Validation via Snapshot Testing}

Idempotence is validated by comparing snapshots:

\begin{lstlisting}[language=Rust]
#[test]
fn test_idempotence_snapshot() {
    let fixture = TestFixture::new()?;

    // Run test and snapshot result
    let result1 = run_test(&fixture)?;
    insta::assert_snapshot!("test_result_1", result1);

    // Run same test again
    let result2 = run_test(&fixture)?;
    insta::assert_snapshot!("test_result_1", result2);

    // Snapshots must be identical
}
\end{lstlisting}

\section{Property 3: Type Preservation}

\begin{theorem}[Type Safety Through Test Lifecycle]
For all test data $D$ of type $T$, operations on $D$ preserve type information:

\begin{equation}
D : T \implies \text{transform}(D) : T'
\end{equation}

where $T'$ is the expected output type.
\end{theorem}

\subsection{Implementation: Generic Fixtures}

Type preservation is enforced through generic fixtures:

\begin{lstlisting}[language=Rust]
// Generic fixture over data type T
pub struct DataFixture<T> {
    data: T,
}

impl<T> TestFixture for DataFixture<T>
where
    T: Default + Clone,
{
    fn new() -> Result<Self, Error> {
        Ok(DataFixture {
            data: T::default(),
        })
    }
}

// Usage: type is preserved throughout
let fixture: DataFixture<Order> = DataFixture::new()?;
let order: Order = fixture.data.clone();  // Type is Order, not generic

// If we try to assign to wrong type, compiler error:
// let customer: Customer = fixture.data;  // ERROR: cannot assign Order to Customer
\end{lstlisting}

\section{Property 4: Boundedness}

\begin{theorem}[Bounded Test Execution]
All test operations complete within measurable time bounds:

\begin{equation}
t(\text{test}) \leq \text{SLO}
\end{equation}

where SLO is specified per test (e.g., 500ms for unit tests).
\end{theorem}

\subsection{Implementation: Performance Tests}

Chicago-tdd-tools provides performance measurement via RDTSC on x86_64:

\begin{lstlisting}[language=Rust]
use chicago_tdd_tools::performance::TickCounter;

#[test]
fn test_within_tick_budget() {
    let counter = TickCounter::now();

    // Execute operation
    let result = expensive_operation();
    assert_ok!(result);

    // Check tick budget
    let ticks = counter.elapsed();
    assert!(ticks <= 1000, "Exceeded tick budget: {} > 1000", ticks);
}
\end{lstlisting}

Performance is measured in CPU ticks (hardware cycles), providing sub-microsecond precision.

\section{Integration: How the Framework Realizes A = µ(O)}

The Chatman Equation is realized through the entire test lifecycle:

\subsection{Step 1: Observations (Test Data)}

Test data represents observations $O$:

\begin{lstlisting}[language=Rust]
// Observations: structured test data
let order = OrderBuilder::new()
    .id(123)
    .amount(99.99)
    .status(OrderStatus::Pending)
    .build()?;  // Type-safe construction
\end{lstlisting}

\subsection{Step 2: Measurement Function (Test Execution)}

The test function $\mu$ operates deterministically on observations:

\begin{lstlisting}[language=Rust]
// Measurement function: pure test logic
fn test_order_processing(order: Order) -> Result<TestResult, Error> {
    // Arrange: setup
    let processor = OrderProcessor::new();

    // Act: execute
    let result = processor.process(&order)?;

    // Assert: verify
    assert_eq!(result.status, OrderStatus::Processed);

    Ok(TestResult::Pass)
}
\end{lstlisting}

\subsection{Step 3: Actions (Test Results)**

Test results are deterministic outputs:

\begin{lstlisting}[language=Rust]
// Actions: test result (Pass/Fail)
let result: TestResult = test_order_processing(order)?;
// Result is deterministic function of input

// Snapshot for reproducibility
insta::assert_snapshot!("order_result", result);
\end{lstlisting}

\section{Validation: Proving the Equation Holds}

\subsection{Empirical Validation}

The framework includes tests that prove $A = \mu(O)$:

\begin{lstlisting}[language=Rust]
#[test]
fn validate_chatman_equation() {
    // Given: observation O
    let obs = create_test_observation();

    // Compute: A1 = µ(O)
    let action1 = measure(&obs)?;

    // Recompute: A2 = µ(O)
    let action2 = measure(&obs)?;

    // Verify: A1 == A2 (determinism)
    assert_eq!(action1, action2);

    // Verify: applying µ twice = applying once (idempotence)
    let action3 = measure(&action1)?;
    assert_eq!(action1, action3);

    // Verify: snapshot consistency
    insta::assert_snapshot!("measurement_result", action1);
}
\end{lstlisting}

\subsection{Property-Based Validation}

Properties are tested across all possible inputs (within the search space):

\begin{lstlisting}[language=Rust]
#[test]
fn property_determinism() {
    proptest!(|(obs in arb_observation())| {
        let result1 = measure(&obs)?;
        let result2 = measure(&obs)?;
        prop_assert_eq!(result1, result2);
    });
}

#[test]
fn property_idempotence() {
    proptest!(|(obs in arb_observation())| {
        let action1 = measure(&obs)?;
        let action2 = measure(&action1)?;
        prop_assert_eq!(action1, action2);
    });
}

#[test]
fn property_type_preservation() {
    proptest!(|(obs in arb_observation())| {
        let action = measure(&obs)?;
        // Type system already proves this, but we validate
        assert!(action_has_correct_type(&action));
    });
}
\end{lstlisting}

\section{The Complete Picture}

Chicago-tdd-tools proves the Chatman Equation through:

\begin{table}[H]
\centering
\caption{Chatman Equation Realization in Testing}
\begin{tabular}{|l|l|l|}
\hline
\textbf{Equation Component} & \textbf{Implementation} & \textbf{Validation} \\
\hline
Observations $O$ & Test fixtures + data builders & Type system \\
\hline
Measurement function $\mu$ & Test logic (AAA pattern) & Determinism tests \\
\hline
Actions $A$ & Test results + snapshots & Snapshot testing \\
\hline
Determinism & Pure functions + no side effects & Property tests \\
\hline
Idempotence & Immutable fixtures + fresh state & Multiple runs \\
\hline
Type preservation & Generic fixtures + type system & Compiler proof \\
\hline
Boundedness & RDTSC tick measurement & Performance tests \\
\hline
\end{tabular}
\end{table}

\section{Conclusion: Tests Are Deterministic by Design}

The framework makes deterministic, reproducible, verifiable tests the default. Tests written
with chicago-tdd-tools:

\begin{enumerate}
  \item Cannot violate the AAA pattern (type system enforces it)
  \item Cannot have invalid states (impossible states are unrepresentable)
  \item Cannot panic unpredictably (error handling is required)
  \item Cannot have global state pollution (fixtures are isolated)
  \item Cannot have flaky behavior (deterministic by construction)
\end{enumerate}

This embodies the core principle of the Chatman Equation: actions are deterministic functions
of observations, measured and verified at every step.

\include{chapters/08-practical-guide}

\backmatter

\bibliographystyle{plainnat}
\bibliography{references}

\appendix

\include{appendix/A-api-reference}
\include{appendix/B-macro-reference}
\include{appendix/C-examples}
\include{appendix/D-cargo-features}

\end{document}
